\begin{frame}{Summary 1-1}

  \large\textbf{Model}
  
  \begin{itemize}[leftmargin=0.6cm]
    \large
    \itemsep6pt
  \item[--] anisotropy in spatial connectivity as a result of
    stereotypical axon and dendrite morphology
  \item[--] anisotropic network as a simple model to test how
    anisotropy in spatial connectivity impacts network connectivity
  \end{itemize}

  \vfill

  \large\textbf{Results}
  
  \begin{itemize}[leftmargin=0.6cm]
    \large
    \itemsep6pt
  \item[--] reciprocal connections are overrepresented but only
    due induced distance-dependency
  \item[--] various neuron properties (such as functional
    similarity) may compound with distance\--dependency to induce
    observed reciprocity in cortical circuits
    
  \end{itemize}


\end{frame}

\begin{frame}{Summary 1-2}

  \large\textbf{Results}
  
  \begin{itemize}[leftmargin=0.6cm]
    \large
    \itemsep6pt
  \item[--] specific three neuron motifs occur over- and
    underrepresented due to anisotropy matching data from
    cortical circuits
  \item[--] observed frequent occurrence of high connection counts in
    neuron groups as a direct result of anisotropy in spatial
    connectivity
    
  \end{itemize}

  \vfill
  
  \large\textbf{Predictions}
  
  \begin{itemize}[leftmargin=0.6cm]
    \large
    \itemsep6pt
  \item[--] anisotropic network model predicts broad distributions of common
    inputs for neuron pairs
  \item[--] anisotropic network model predicts stronger sensitivity to
    connection type for common input distributions
  \end{itemize}  
  
\end{frame}
