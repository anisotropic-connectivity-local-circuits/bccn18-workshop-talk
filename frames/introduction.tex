\begin{frame}{The dynamic connectome}

  \Large
  At the synapse level, brain circuitry is constantly making new connections and abolishing old ones \nocite{Rumpel2016}

  \vspace{0.4cm}
  \only<2->{\vspace{-0.15cm}}
  
  \begin{figure}
    \centering
    \includegraphics<1>[height=0.5\textheight]{%
      figures/Loewenstein2015_Fig1A.png} %
    \includegraphics<2->[height=0.5\textheight]{%
      figures/Loewenstein2015_Fig1B.png} %
    \includegraphics<2->[height=0.5\textheight]{%
      figures/Loewenstein2015_Fig2A.png} %
  \end{figure}
  
  \source{\cite{Loewenstein2015}}

  \pnote{

    - chronic two-photon imaging through\\
    a cranial window
    
    - auditory cortex in mice (expressing GFP\\
    in a subset of pyramidal neurons)

    Figure: 4 Days apart!

    - image spines from 8 neurons\\
    spine count almost constant!
       
    
  }
  
\end{frame}


\begin{frame}{Robust nonrandom connectivity patterns}

  
  \begin{figure}
    \centering
    \includegraphics<1>[height=0.68\textheight]{%
      figures/Song2005_Fig2.png} %    
    \includegraphics<2>[height=0.7\textheight]{%
      figures/Song2005_Fig4B.png} %
    \includegraphics<3>[height=0.7\textheight]{%
      figures/Perin2011_Fig2AB_select.jpg} %
  \end{figure}


  
  \Large
  \only<1>{\vspace{0.1cm}\begin{center}
      \textbf{1.} Overrepresentation of reciprocal connections
    \end{center}}
  \only<2>{
    \begin{center}
      \textbf{2.} Characteristic occurrence of triplet motifs
    \end{center}}
  \only<3>{
    \begin{center}
      \textbf{3.} High degree of connectivity in clusters occurs frequently
    \end{center}}


  
\only<1>{\source{\cite{Song2005}}}
\only<2>{\source{\cite{Song2005, Perin2011}}}
\only<3>{\source{\cite{Perin2011}}}

  \pnote{

    
  }
  
\end{frame}