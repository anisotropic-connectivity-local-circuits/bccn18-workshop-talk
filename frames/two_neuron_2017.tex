\begin{frame}{}

  \setlength{\parindent}{0pt}

  \vspace{-0.2cm}

  \begin{tabular}{  p{0.43\textwidth}   p{0.02\textwidth}  p{0.43\textwidth} }

    \onslide<1->{\only<1-7>{
    \begin{center}
      \mbox{\myul{\textit{\textbf{Standard random network model}}}}
    \end{center}}}       

    &&
    \onslide<4->{
    \begin{center}
      \mbox{\myul{\textit{\textbf{Varying connection probabilities}}}}
    \end{center}}

    \\
    % 
    \onslide<1->{\only<1-7>{%
    \vspace{-0.8cm}\begin{center}%
      \includegraphics[width=0.69\linewidth]{%
        figures/tikz_line.pdf} %
    \end{center}}%
        \only<8-10>{\begin{figure}%
            \centering\vspace{-0.6cm}
          \includegraphics[width=0.2\textwidth]{%
          figures/pair_xkxl.pdf} %
      \end{figure}}
        }


    &&

    \onslide<4->{
    \begin{center}\vspace{-0.71cm}
      \includegraphics[width=0.69\linewidth]{%
        figures/tikz_line_right.pdf} % 
    \end{center}%\vspace{1cm}
       }
    \\
    \onslide<1->{   \vspace{-0.27cm} 
    Probability of connection a \myul{constant} $P_{ij}$,
    %\vspace{0.05cm}

    \begin{align*}
      P_{ij} = c
    \end{align*}
    }
    &&

    \onslide<5->{\vspace{-0.27cm}
    Probability of connection a \myul{random variable} $P_{ij}$,
    \vspace{0.01cm}
    
    \begin{align*}
      \mathbf{Prob}(P_{ij}=x_k) = F(x_k)
    \end{align*}}	

    \\

    \onslide<2->{\vspace{-0.88cm}
    \textbf{Overall connection probability}
    \vspace{0.12cm}
    
    \begin{align*}
      \mu = P_{ij} = c
    \end{align*}}


    &&
    \onslide<6->{\vspace{-0.88cm}
    \textbf{Overall connection probability}
    \vspace{-0.08cm}
    
    \begin{align*}
      \mu = \sum_{k=1}^m F(x_k) x_k
    \end{align*}}
    
    \\
    \onslide<3->{\vspace{-0.8cm}
    \textbf{Bidirectional connection}
    \vspace{0.12cm}
    
    \begin{align*}
      P_{\text{bidir}} = P_{ij} P_{ji} = c^2
    \end{align*}	}

    &&
    \onslide<7->{\vspace{-0.8cm}
    \textbf{Bidirectional connection}
    \vspace{-0.08cm}

       \only<7-8>{\vspace{0.18cm}
    \begin{align*}
      P_{\text{bidir}} = \text{?} %\label{eq:TT}
    \end{align*}}
       \only<9-10>{
    \begin{align*}
      P_{\text{bidir}} = \sum_{k=1}^m \sum_{l=1}^m F(x_k) x_k %
      \only<9>{F(x_l | x_k)}%
      \only<10>{\myul[red]{F(x_l | x_k)}} x_l 
    \end{align*}}}
    
  \end{tabular}

  
  % \source{\cite{Hoffmann2017}}
  
\end{frame}




\begin{frame}{}

    % 
    \begin{columns}
      % 
      \begin{column}{.5\textwidth}
        \minipage[c][0.85\textheight][s]{\columnwidth}

        \onslide<2->
        \begin{align*}
          P_{\text{bidir}} &= \sum_{k=1}^m \sum_{l=1}^m F(x_k) x_k %
                             F(x_l | x_k) x_l \\ %
                           &= \sum_{k=1}^m F(x_k) x_k^2 .
        \end{align*}


        \onslide<3->
        \textit{Relative overrepresentation} $\varrho$ is the fraction

        \begin{align*}
          \varrho = \frac{P_{\text{bidir}}}{\mu^2} = %
          \frac{\sum_{k=1}^m F(x_k) x_k^2 }%
               {\left(\sum_{k=1}^m F(x_k) x_k\right)^2}.
        \end{align*}

        \onslide<4->
        By Jensen's inequality,

        \begin{align*}
          \left(\sum_{k=1}^m F(x_k) x_k\right)^2 \leq %
          \sum_{k=1}^m F(x_k) x_k^2  \quad \text{and thus}%
          \quad \varrho \geq 1.
        \end{align*}	

        
        \endminipage      
      \end{column}
      % 
      \begin{column}{.5\textwidth}
        \onslide<1->
        \minipage[c][0.85\textheight][s]{\columnwidth}

        \onslide<1->
        \begin{align*}
          F(x_l | x_k) = %
          \begin{cases}%
            1 & \text{if $l = k$} \\%
            0 & \text{otherwise.}
          \end{cases}
        \end{align*}

      
        \begin{figure}
          \centering
          \includegraphics[width=0.85\textwidth]{%
            figures/two_point_network.pdf} %
        \end{figure}

        \vspace{1.25cm}
        
        \endminipage
                
      \end{column}
      
    \end{columns}
    
    \onslide<1>\source{\normalsize \cite{Hoffmann2017}}

\end{frame}



\begin{frame}{}

  \begin{figure}
    \centering
    \includegraphics[width=0.8\textwidth]{%
      figures/nrnd_gamma.png} %
  \end{figure}

  \vspace{0.2cm}

  \begin{itemize}[leftmargin=1.3cm]
    \large
    \itemsep9pt
    \item[--]<2-> multiple neuron properties together can cause %
                  strong overrepresentation
    \item[--]<3> example: higher connection probability in functionally
                 \\ related cells \parencite{Lee2016a}       
  \end{itemize}

  \vspace{0.4cm}

  \source{\normalsize \cite{Hoffmann2017}}
  
\end{frame}




